\section{ WebDriver Exception : Can not connect to the service geckodriver}
Cara mengatasinya yaitu dengan menyamakan versi python, mozilla firefox, geckodriver , misalnya versi 64bit. Jika masih tidak bisa, silahkan buka localhost pada laptop atau komputer anda. Lokasi localhost berada di (C: Windows System32 drivers etc) kemudian buka file hosts dan edit menggunakan notepad++ kemungkian error karena di hosts 127.0.0.1


\section{SessionNotCreatedException : Unable to find a matching set of capabilities}
Cara mengatasinya yaitu dengan pada desktop klik kanan pada mozilla firefox, pilih properties dan copy pada bagian lokasi direktori Jika sudah paste kedalam kode program pada binnary mozilla firefox


\section{Module Notfound Error : no module named selenium}
Cara mengatasinya:
\begin{enumerate}
\item  uninstall aplikasi berbahasa pemrograman python, selain anaconda karena nanti akan terjadi konflik
\item restart komputer anda 
\item setting environtment kembali
\item buka cmd
\item install kembali selenium 
\item selesai
\end{enumerate} 

\section{ WebDriverExecption: cant load profile Possible version mismatch}
Cara mengatasinya cukup mengupdate versi Mozilla Firefox atau Chrome agar bisa compatible dengan drivernya.
